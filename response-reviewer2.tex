\documentclass{article}
\begin{document}

\title{Response Reviewer 2}
\maketitle

The author would like to thank the reviewer for the various comments
and suggestions which greatly helped in improving the quality of the manuscript. 

In the proceeding section they will find the revisions/responses for
each concern outlined. The tracked changes are highlighted in the file \emph{differences.pdf}.

\section*{Draft Comments \& Revisions}

\begin{enumerate}

\item{

\fbox{%
    \parbox{\textwidth}{%
      \textbf{Comment}: Within the abstract the word ‘that’ is missing
      from “We highlight design decisions and practical issues that
      must be addressed...”

    }
  }

\textbf{Response}: Modifications done.

}

\item{

\fbox{%
    \parbox{\textwidth}{%
      \textbf{Comment}: When describing the magnetosphere, saying that
      Earth’s field “dominates the magnetic field of outer space” is
      vague. “Outer space” could be defined as outside the
      heliosphere. Within the heliosphere the sun’s magnetic field is
      dominant except for close to planets with their own planetary magnetospheres.
    }
  }

\textbf{Response}: Changes made in section 1 on page 1.

}

\item{

\fbox{%
    \parbox{\textwidth}{%
      \textbf{Comment}: I would suggest replacing “some popular
      indices are” with “some common indices are.”
    }
  }

\textbf{Response}: Changes done.

}

\item{

\fbox{%
    \parbox{\textwidth}{%
      \textbf{Comment}: Typically we would say “quiet day curve”
      instead of “calm day curve”
    }
  }

\textbf{Response}: Changes made on page 1.

}

\item{

\fbox{%
    \parbox{\textwidth}{%
      \textbf{Comment}: Ionosphere does not need to be capitalized in
      the description of the AE index
    }
  }

\textbf{Response}: Changes made on page 2.

}

\item{

\fbox{%
    \parbox{\textwidth}{%
      \textbf{Comment}: Reword the Dst section as the first sentence
      is confusion and difficult to read. Perhaps omitting “weakening
      or strengthening” and just discussing how Dst measures the
      strength of the ring current.
    }
  }

\textbf{Response}: Changes made on page 2.

}

\item{

\fbox{%
    \parbox{\textwidth}{%
      \textbf{Comment}: A geomagnetic storm with a Dst of ≤ 150 nT
      would be quiet large. Typically smaller values are used to
      indicate the occurrence of the geomagnetic storm (value of ≤ 100
      nT or so). It might be good to include a reference here.
    }
  }

\textbf{Response}: Changes made on page 2.

}

\item{

\fbox{%
    \parbox{\textwidth}{%
      \textbf{Comment}: The space weather effects of geomagnetic
      storms are much greater than just “their effects on
      telecommunications infrastructure”. There are lots of other
      effects that should be mentioned as well: the effects of the
      power grid, satellites, GPS/GNSS, etc.
    }
  }

\textbf{Response}: Modifications made on page 2.

}

\item{

\fbox{%
    \parbox{\textwidth}{%
      \textbf{Comment}: “The principal data source for forecasting of
      geomagnetic indices is the OMNI data set hosted by NASA...” This
      sentence is wrong. For Kp ground-based magnetometers are used
      and for Dst real- time data from L1 (from either the DSCOVR or
      ACE satellite) is used. 
    }
  }

\textbf{Response}: Modifications made on page 2 section 1.

}

\item{

\fbox{%
    \parbox{\textwidth}{%
      \textbf{Comment}: The data source for real-time forecasting is
      not OMNI. It is fine to discuss the OMNI data since that is what
      is used later in the chapter for the example study, but it might
      be better to discuss the OMNI data when working through the
      example so as not to confuse the reader about real- time data
      sources used for real-time space weather forecasting.
    }
  }

\textbf{Response}: It is mentioned on page 2, section 1.

}

\item{

\fbox{%
    \parbox{\textwidth}{%
      \textbf{Comment}: If you are only using 1-hour values, how can
      you get a useful real-time prediction?
    }
  }

\textbf{Response}: The question is not relevant to the scope of the
chapter, if one is interested in making predictions for geomagnetic
indices minutes in the future then one may apply the methodology
outlined to minute cadence data.

}

\item{

\fbox{%
    \parbox{\textwidth}{%
      \textbf{Comment}: Please define what m and K are within the text.
    }
  }

\textbf{Response}: Defined on page 4, section 3.1

}

\item{

\fbox{%
    \parbox{\textwidth}{%
      \textbf{Comment}: Please clarify why you are using hourly OMNI
      data to predict Dst. Wouldn’t it be better to use the 1-minute
      data if you want to actually generate a useful forecast?
    }
  }

\textbf{Response}: This question is beyond the scope of this
chapter. The aim of the chapter is to educate practitioners on
applications of the GP methodology for time series prediction, the
choice of data cadence is not of primary relevance here. Depending on
the particular application and research questions being considered,
practitioners would choose which data set is more suitable for their purpose.

}

\item{

\fbox{%
    \parbox{\textwidth}{%
      \textbf{Comment}: When you mention the lag time p, more
      discussion would be helpful. How far back in time do you
      typically need to go? Does it depend more on the amount of time
      or the number of points? If I use 1-min data instead of 1-hour
      data do I need to use more points?
    }
  }

\textbf{Response}: Modifications made in section 5.4. Choice of lag
time p is empirical, how it changes depends on the characteristics of
the particular time series and data set being considered.

}

\item{

\fbox{%
    \parbox{\textwidth}{%
      \textbf{Comment}: Be consistent in either writing “sub-storm” or “substorm”
    }
  }

\textbf{Response}: Modifications made in the manuscript.

}

\item{

\fbox{%
    \parbox{\textwidth}{%
      \textbf{Comment}: Define RBE within the text in addition to in the table.
    }
  }

\textbf{Response}: RBF has been defined on page 7, section 5.2.

}

\item{

\fbox{%
    \parbox{\textwidth}{%
      \textbf{Comment}: Hyper-parameters is not hyphenated in the
      table while it is elsewhere within the text.
    }
  }

\textbf{Response}: Modifications made throughout the manuscript.

}

\item{

\fbox{%
    \parbox{\textwidth}{%
      \textbf{Comment}: Define Q in the text when it is first discussed.
    }
  }

\textbf{Response}: Changes made in section 5.3.

}

\item{

\fbox{%
    \parbox{\textwidth}{%
      \textbf{Comment}: In section 6.3 a more detailed description of
      how you got to the last equation would be useful (as would
      equation numbers).
    }
  }

\textbf{Response}: The equation is nothing but the logarithm of the
density function of the multivariate gaussian which is a very well
known equation in mathematics and science. (Section 5.3 page 8)

}

\item{

\fbox{%
    \parbox{\textwidth}{%
      \textbf{Comment}: “Some of the techniques used for model
      selection in the context of GPR include.” This is a fragment of
      a sentence.
    }
  }

\textbf{Response}: Changes made on page 9.

}

\item{

\fbox{%
    \parbox{\textwidth}{%
      \textbf{Comment}: In the first paragraph of section 6.4, please
      explain why it would not be beneficial to group the quantities
      with the kernel hyper-parameters in the model selection procedure.
    }
  }

\textbf{Response}: Modifications made in section 5.4 on page 10.

}

\item{

\fbox{%
    \parbox{\textwidth}{%
      \textbf{Comment}: Why does Table 2 come after Tables 3 and 4 in
      the text?
    }
  }

\textbf{Response}: Appropriate corrections made in the manuscript.

}

\item{

\fbox{%
    \parbox{\textwidth}{%
      \textbf{Comment}: When choosing performance metrics, there are
      several other metrics that are traditionally used by the weather
      forecasting community that would be good to include such as the
      Heidke Skill Score (HSS). A great reference is Statistical
      Methods in Atmospheric Science by Daniel Wilks.
    }
  }

\textbf{Response}: The Heikde Skill Score is a metric used for
categorical predictive problems, a reference to the aforementioned
book has been added in section 6 point 3 on page 11.

}

\item{

\fbox{%
    \parbox{\textwidth}{%
      \textbf{Comment}: When discussing the maximum time lag, use
      units (min or hour) and discuss how these results would change
      if using minutes vs. hours.
    }
  }

\textbf{Response}: It is not possible to rigourously state how the
maximum time lag would change if using minutes instead of hours, as it
is empirically determined. That is why the chapter discusses a
workflow which can be applied in either case.

}

\item{

\fbox{%
    \parbox{\textwidth}{%
      \textbf{Comment}: Please elaborate on how the decision is made
      as to which model order yields the “best performance”.
    }
  }

\textbf{Response}: Changes have been made in section 6 point 4.

}

\item{

\fbox{%
    \parbox{\textwidth}{%
      \textbf{Comment}: This section could benefit from a more though
      discussion. As this is a text book, I think it would be useful
      for students if a more comprehensive description was included.
    }
  }

\textbf{Response}: Changes have been made to section 9.

}

\item{

\fbox{%
    \parbox{\textwidth}{%
      \textbf{Comment}: What did the persistence model predict? Could
      you plot that on the graph as well? A discussion of the
      persistence forecast results compared to these in the text would
      be helpful.
    }
  }

\textbf{Response}: Persistence model simply predicts the Dst value of
the previous hour as the prediction for the current hour. It is simply
a baseline predictor which can be used for simple evaluation of
predictive models.

}

\item{

\fbox{%
    \parbox{\textwidth}{%
      \textbf{Comment}: How much lead time do you get with this method? 
    }
  }

\textbf{Response}: Since the example is of one hour ahead prediction,
it gives a lead time of one hour.

}

\item{

\fbox{%
    \parbox{\textwidth}{%
      \textbf{Comment}: The chapter just ends. A conclusion or summary
      would be helpful.
    }
  }

\textbf{Response}: A conclusion has been added in section 9.

}

\end{enumerate}


\end{document}