\documentclass{report}

\usepackage{amsmath}
\usepackage{txfonts}
\usepackage{graphicx}
\usepackage{booktabs}
\usepackage{url}

\title{Probabilistic Forecasting of the Disturbance Storm Time Index: An Autoregressive Gaussian Process approach}

\begin{document}


\begin{abstract}
We present a methodology for generating probabilistic predictions for the \emph{Disturbance Storm Time} ($Dst$) geomagnetic activity index. We focus on the \emph{One Step Ahead} (OSA) prediction task and use the OMNI hourly resolution data to build our models.

Our proposed methodology is based on the technique of \emph{Gaussian Process Regression} (GPR). Within this framework we develop two models; \emph{Gaussian Process Auto-Regressive} (GP-AR) and \emph{Gaussian Process Auto-Regressive with eXogenous inputs} (GP-ARX). 

We also propose a criterion to aid model selection with respect to the order of auto-regressive inputs. Finally we test the performance of the GP-AR and GP-ARX models on a set of 63 geomagnetic storms between 1998 and 2006 and illustrate sample predictions with error bars for some of these events.

\end{abstract}


\section{Intro}

\end{document}