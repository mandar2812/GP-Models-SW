\documentclass{article}

\usepackage{amsmath}
\usepackage{txfonts}
\usepackage{graphicx}
\usepackage{booktabs}
\usepackage{url}
\usepackage{natbib}
\bibliographystyle{biblio.bst}

\title{Probabilistic Forecasting of Geomagnetic Indices using Gaussian Process Models}

\begin{document}


\begin{abstract}

In this chapter, we give the reader an indepth view into building of probabilistic forecasting models for geomagnetic time series using the \emph{Gaussian Process} methodology outlined in the previous chapters. 

\end{abstract}

\section{Geomagnetic Time Series \& Forecasting}

The Earth's magnetosphere is a region around the planet where the Earth's own magnetic field dominates the magnetic field of outer space. It is a region which is impinged upon constantly by the solar wind. The ionised plasma ejected by the sun couples with the Earth's magnetic field in a complex manner leading to highly non-linear and chaotic processes which determine the state of the magnetosphere. It is quite common to reduce these complex dependencies by condensing the state of the Earth's magnetosphere into a set of geomagnetic indices.

Geomagnetic indices come in various forms, they may take continuous or discrete values and may be defined with varying time resolutions. Their values are often calculated by averaging or combining a number of readings taken by instruments, usually magnetometers, around the Earth. Each geomagnetic index is a proxy for a particular kind of phenomenon. Some popular indices are the $K_p$, $Dst$ and the $AE$ index.

\begin{enumerate}
    \item $K_p$: The Kp-index is a discrete valued global geomagnetic activity index and is based on 3 hour measurements of the K-indices \citep{Bartels}. The K-index itself is a three hour long quasi-logarithmic local index of the geomagnetic activity, relative to a calm day curve for the given location.
    
    \item $AE$: The Auroral Electrojet Index, $AE$, is designed to provide a global, quantitative measure of auroral zone magnetic activity produced by enhanced Ionospheric currents flowing below and within the auroral oval \citep{AEIndex}. It is a continuous index which is calculated every hour.
    
    \item $Dst$: A continuous hourly index which gives a measure of the weakening or strengthening of the Earth's equatorial magnetic field due to the weakening or strengthening of the ring currents and the geomagnetic storms \citep{DesslerAndParker}. 
\end{enumerate}

Space weather forecasting systems usually use in-situ measurements of solar wind parameters, taken by satellites, as well as historical data of indices to produce forcasts for various geomagnetic time series. In this chapter for the purpose of exposition, we will focus on one hour ahead prediction of the $Dst$ time series. 

A number of modeling techniques have been applied for the prediction of the $Dst$ index. One of the earliest forecasting techniques involves calculating the $Dst(t)$ as a solution of an \emph{Ordinary Differential Equation} (ODE) which expressed the rate of change of $Dst(t)$ as a combination of two terms: decay and injection $\frac{d Dst(t)}{dt} = Q(t) - \frac{Dst(t)}{\tau}$, where $Q(t)$ relates to the particle injection from the plasma sheet into the inner magnetosphere. This method was presented first by \citet{JGR:JGR10260} and later modified and extended in works such as \citet{Wang:Dst}, \citet{JGRA:JGRA14856}, \citet{Ballatore2014} and others.

%Talk about NARMAX Dst

Important empirical geomagnetic prediction models include the \emph{Nonlinear Auto-Regessive Moving Average with eXogenous inputs} (NARMAX) methodology (see \citet{doi:10.1080/00207178908559767}, \citet{GRL:GRL13494}, \citet{GRL:GRL20944}, \citet{JGRA:JGRA18657}, \citet{balikhin:narmax}, \citet{JGRA:JGRA20661}, \citet{JGRA:JGRA50192}) and \emph{Artificial Neural Networks} (ANN) based models (\citet{Lund}, \citet{JGRA:JGRA17461}, \citet{SWE:SWE286}, \citet{pallocchia:hal-00318011}) for time series prediction of $Dst$ and $Kp$ indices from interplanetary magnetic field data and solar wind parameters. 

%Talk about need for probabilistic forecasts.
Although much research has been done on prediction of the $Dst$ index, much less has been done on probabilistic forecasting of $Dst$. One such work described in \citet{McPherron:2013} involves identification of high speed solar wind streams using the WSA model, using predictions of high speed streams to construct ensembles of $Dst$ trajectories which yield the quartiles of $Dst$ time series. 

Probabilistic forecasting is of particular importance in geophysics applications as the end users of forecasts often require confidence bounds on the said forecasts. It is in this context where \emph{Gaussian Processes} become especially attractive due to their inherent probabilistic formulation and tractability of analytical inference.


\section{Gaussian Processes}

\emph{Gaussian Processes} first appeared in machine learning research in \citet{Neal:1996:BLN:525544}, as the limiting case of Bayesian inference performed on neural networks with infinitely many neurons in the hidden layers. Although their inception in the machine learning community is recent, their origins can be traced back to the geo-statistics research community where they are known as \emph{Kriging} methods \citep{krige1951statistical}. In pure Mathematics, \emph{Gaussian Processes} have been studied extensively and their existence was first proven by Kolmogorov's extension theorem \citep{tao2011introduction}. The reader is referred to \citet{Rasmussen:2005:GPM:1162254} for an in depth treatment of Gaussian Processes in machine learning.

Without going into too many details, we give a quick recap of the formulation and exact inference in \emph{Gaussian Process Regression} (GPR) models. 

\subsection{Gaussian Process Regression: Formulation}

Our aim is to infer an unknown function $f(\mathbf{x})$ from its noise corrupted measurements $(\mathbf{x}_i, y_i)$ where $y_i = f(\mathbf{x}_i) + \epsilon$ and $\epsilon \sim \mathcal{N}(0, \sigma^2)$ is independent and identically distributed Gaussian noise.

A \emph{Gaussian Process} model represents the finite dimensional probability distribution of $f(\mathbf{x}_i)$ by a multivariate gaussian having a particular structure for its mean and covariance as shown in \ref{eq:normal} - \ref{eq:sto}.

\begin{align}
 \mathbf{f} = & \left( \begin{array}{c} f(\mathbf{x}_1) \\ f(\mathbf{x}_2) \\ \vdots \\ f(\mathbf{x}_N) \end{array} \right) \label{eq:fvalues}\\
 \vspace{2\baselineskip}
 \mathbf{f} | \mathbf{x}_1, \cdots, \mathbf{x}_N \sim & \mathcal{N}\left( \mathbf{m}, \mathbf{K} \right)  \label{eq:normal}\\
 \vspace{2\baselineskip}
 \mathbb{P}( \mathbf{f} \ | \ \mathbf{x}_1, \cdots, \mathbf{x}_N) = & \frac{1}{(2\pi)^{n/2} det(\mathbf{K})^{1/2}} exp \left(-\frac{1}{2} (\mathbf{f} - \mathbf{m})^T \mathbf{K}^{-1} (\mathbf{f} - \mathbf{m}) \right) \label{eq:sto}
\end{align}

In order to uniquely define the distribution of $\mathbf{f}$, it is required to specify $\mathbf{m}$ and $\mathbf{K}$. For this probability density to be valid, there are further requirements imposed on $\mathbf{K}$: 

\begin{enumerate}
      \item Symmetry: $\mathbf{K}_{ij} = \mathbf{K}_{ji} \ \forall i,j \in {1, \cdots, N} $ 
      \item Positive Semi-definiteness: $\mathbf{z}^T \mathbf{K} \mathbf{z} \geq 0 \ \forall \mathbf{z} \in \mathbb{R}^N$  
\end{enumerate}

In \emph{Gaussian Processes} the individual elements of $\mathbf{x}$ and $\mathbf{K}$ are specified in the form of functions as shown below.

\begin{align}
      \mu_i = & \mathbb{E}[f(\mathbf{x}_i)] := m(\mathbf{x}_i) \\
      \Lambda_{ij} = & \mathbb{E}[(f(\mathbf{x}_i) - \mu_i)(f(\mathbf{x}_j) - \mu_j)] := K(\mathbf{x}_i, \mathbf{x}_j)
\end{align}

In the machine learning community, $m(.)$ and $K(.,.)$ are known as the \emph{mean function} and \emph{covariance function} or \emph{kernel function} of the process respectively. Giving a closed form expression for $m(.)$ and $K(.,.)$ uniquely specifies a particular \emph{Gaussian Process} and so a GP model is often expressed notationally as shown below.

\begin{equation}
    f(\mathbf{x}) \sim \mathcal{GP}(m(\mathbf{x}), K(\mathbf{x}, \mathbf{x}'))
\end{equation}

\subsection{Gaussian Process Regression: Inference}

In order to generate predictions $f(\mathbf{x}^{*}_i)$ for a set of test points $ {\mathbf{x}^{*}_i : \forall i \in 1, \cdots, M} $. Using the multivariate Gaussian distribution in equation (\ref{eq:sto}) we can construct the joint distribution of $f(\mathbf{x})$ over the training and test points.

\begin{align}
    \mathbf{f}_* = & \left( \begin{array}{c} f(\mathbf{x^{*}_1}) \\ f(\mathbf{x^{*}_2}) \\ \vdots \\ f(\mathbf{x^{*}_M}) \end{array} \right)_{M \times 1} \\
     \vspace{4\baselineskip}
    \left( \begin{array}{c} \mathbf{y} \\ \mathbf{f_*} \end{array} \right) | \ \ \mathbf{X}, \mathbf{X}_* \sim & 
    \mathcal{N}\left(\mathbf{0}, \left[ \begin{array}{cc} \mathbf{K} + \sigma^{2} \mathbf{I} & \mathbf{K}_{*} \\ \mathbf{K}_{*}^T & \mathbf{K}_{**} \end{array} \right ] \right) \label{eq:dist}
\end{align}

Probabilistic predictions $f_*$ can be generated by constructing the conditional distribution $\mathbf{f_*}|\mathbf{X},\mathbf{y},\mathbf{X_*}$ which is also a multivariate gaussian as shown in equation \ref{eq:posterior}.

\begin{align}
    & \mathbf{f_*}|\mathbf{X},\mathbf{y},\mathbf{X_*} \sim \mathcal{N}(\mathbf{\bar{f}_*}, \Sigma_*)  \label{eq:posterior} \\
    & \mathbf{\bar{f}_*} =  \ \mathbf{K}^T_{*} [\mathbf{K} + \sigma^{2} \mathbf{I}]^{-1} \mathbf{y} \label{eq:posteriormean} \\
    & \Sigma_* = \ \mathbf{K}_{**} - \mathbf{K}^T_{*} \left(\mathbf{K} + \sigma^{2} \mathbf{I}\right)^{-1} \mathbf{K}_{*} \label{eq:posteriorcov}
\end{align}

\section{One Hour Ahead $Dst$: Formulation} \label{sec:osa}

Below in equations (\ref{eq:Dst}) - (\ref{eq:DstGP}) we outline a \emph{Gaussian Process} formulation for \emph{OSA} prediction of $Dst$. A vector of features $\mathbf{x}_{t-1}$ is used as input to an unknown function $f(\mathbf{x}_{t-1})$.

The features $\mathbf{x}_{t-1}$ can be any collection of quantities in the hourly resolution OMNI data set. Generally $\mathbf{x}_{t-1}$ are time histories of $Dst$ and other important variables such as plasma pressure $p(t)$, solar wind speed $V(t)$, $z$ component of the interplanetary magnetic field $B_z(t)$.


\begin{align}
    Dst(t) & =  f(\mathbf{x}_{t-1}) + \epsilon \label{eq:Dst} \\
    \epsilon & \sim  \mathcal{N}(0, \sigma^2) \label{eq:GPNoise} \\
    f(x_t) & \sim  \mathcal{GP}(m(\mathbf{x}_t), K_{osa}(\mathbf{x}_t, \mathbf{x}_s)) \label{eq:DstGP} \\
\end{align}

We consider two choices for the input features $\mathbf{x}_{t-1}$ leading to two variants of \emph{Gaussian Process} regression for $Dst$ time series prediction.

\subsection{Gaussian Process Auto-Regressive (GP-AR)} \label{sec:gpar}

The simplest auto-regressive models for \emph{OSA} prediction of $Dst$ are those that use only the history of $Dst$ to construct input features for model training. The input features $\mathbf{x}_{t-1}$ at each time step are the history of $Dst(t)$ until a time lag of $p$ hours.

\begin{align*}
    \mathbf{x}_{t-1} = & \left(Dst(t-1), \cdots , Dst(t-p+1)\right)
\end{align*}

\subsection{Gaussian Process Auto-Regressive with eXogenous inputs (GP-ARX)} \label{sec:gparx}

Auto-regressive models can be augmented by including exogenous quantities in the inputs $\mathbf{x}_{t-1}$ at each time step, in order to improve predictive accuracy. $Dst$ gives a measure of ring currents, which are modulated by plasma sheet particle injections into the inner magnetosphere during sub-storms. Studies have shown that the substorm occurrence rate increases with solar wind velocity (high speed streams) \citet{Kissinger2011,Newell2016}. Prolonged southward interplanetary magnetic field (IMF) $z$-component ($B_z$) is needed for sub-storms to occur \citet{McPherron1986}. An increase in the solar wind electric field, $V_{sw}B_z$, can increase the dawn-dusk electric field in the magnetotail, which in turn determines the amount of plasma sheet particle that move to the inner magnetosphere \citet{Friedel2001}. Therefore, our exogenous parameters consist of solar wind velocity $V_{sw}$ and IMF $B_z$.   

In this model we choose distinct time lags $p$, $p_{v}$ and $p_{b}$ for $Dst$, $V$ and $B_z$ respectively.
    
\begin{align*}
       \mathbf{x}_{t-1} & = (Dst(t-1), \cdots , Dst(t-p+1), \\
        & \ \ \ \ \  V_{sw}(t-1), \cdots, V_{sw}(t-p_{v}+1),\\
        & \ \ \ \ \  B_{z}(t-1), \cdots, B_{z}(t-p_{b}+1))
\end{align*}

\section{One Hour Ahead $Dst$: Model Design}

\subsection{Choice of Mean Function}

Mean functions in GPR models encode trends in the data, they are the baseline predictions the model falls back to in case the training and test data have little correlation as predicted by the kernel function. If there is no prior knowledge about the function to be approximated, \citet{Rasmussen:2005:GPM:1162254} state that it is perfectly reasonable to choose $m(\mathbf{x} = 0)$ as the mean function, as long as the target values are normalized. In the case of the $Dst$ time series, it is known that the so called \emph{persistence model} $\hat{D}st(t) = Dst(t-1)$ performs quite well in the context of OSA prediction. We therefore choose the \emph{persistence model} as the mean function in our OSA Dst models.

\subsection{Choice of Kernel}

For the success of a \emph{Gaussian Process} model an appropriate choice of kernel function is paramount. The symmetry and positive semi-definiteness of \emph{Gaussian Process} kernels implies that they represent inner-products between some basis function representation of the data. The interested reader is suggested to refer to \citet{Berlinet2004}, \citet{Scholkopf:2001:LKS:559923} and \citet{hofmann2008} for a thorough treatment of kernel functions and the rich theory behind them. Some common kernel functions used in machine learning are listed in Table \ref{table:kernel}.

\begin{table}[h]
\caption{Popular Kernel functions used in GPR models}
\centering
\begin{tabular}{l c c}
\hline
 Name  & Expression & Hyperparameters  \\
\hline
  Radial Basis Function (RBF)  & $\frac{1}{2} exp(-||\mathbf{x} - \mathbf{y}||^2/l^2)$  & $l \in \mathbb{R}$   \\
  
  Polynomial  & $(\mathbf{x}^\intercal \mathbf{y} + b)^d$ & $b \in \mathbb{R}, d \in \mathbb{N}$   \\
  
  Laplacian  & $exp(-||\mathbf{x} - \mathbf{y}||_{1}/\theta)$  & $\theta \in \mathbb{R}^+$  \\
  
  Student's T  & $1/(1 + ||\mathbf{x} - \mathbf{y}||_{2}^d)$ & $d \in \mathbb{R}^{+}$\\
  
  Maximum Likelihood Perceptron  & $sin^{-1}(\frac{w\mathbf{x}^\intercal \mathbf{y} + b}{\sqrt{w\mathbf{x}^\intercal \mathbf{x} + b + 1} \sqrt{w\mathbf{y}^\intercal \mathbf{y} + b + 1}})$ & $w, b \in \mathbb{R}^{+}$\\
\hline
\end{tabular}
\label{table:kernel}
\end{table}

The quantities $l$ in the RBF, and $b$ and $d$ in the polynomial kernel are known as \emph{hyper-parameters}. Hyper-parameters give flexibility to a particular kernel structure, for example $d = 1, 2, 3, \cdots$ in the polynomial kernel represents linear, quadratic, cubic and higher order polynomials respectively. The process of assigning values to the \emph{hyper-parameters} is crucial in the model building process and is known as \emph{model selection}. 

In this text, we construct GPR models with a combination of the \emph{maximum likelihood perceptron} kernel and \emph{student's T} kernel as shown in equations (\ref{eq:usedKernel}). The \emph{maximum likelihood perceptron} kernel is the \emph{Gaussian Process} equivalent of a single hidden layer feed-forward neural network model as demonstrated in \citet{Neal:1996:BLN:525544}.

\begin{align}
    K_{osa}(\mathbf{x}, \mathbf{y}) & = K_{mlp}(\mathbf{x}, \mathbf{y}) + K_{st}(\mathbf{x}, \mathbf{y}) \label{eq:usedKernel} \\
    K_{mlp}(\mathbf{x}, \mathbf{y}) & = sin^{-1}(\frac{w\mathbf{x}^\intercal \mathbf{y} + b}{\sqrt{w\mathbf{x}^\intercal \mathbf{x} + b + 1} \sqrt{w\mathbf{y}^\intercal \mathbf{y} + b + 1}}) \\
    K_{st}(\mathbf{x}, \mathbf{y}) & = \frac{1}{1 + ||\mathbf{x} - \mathbf{y}||_{2}^d}
\end{align}


\subsection{Model Selection}

Given a GPR model with a kernel function $K_\theta$, the problem of model selection consists of finding appropriate values for the kernel hyper-parameters $\theta = \left(\theta_1, \theta_2, \cdots, \theta_i\right)$. In order to assign a value to $\theta$, we must define an objective function which represents our confidence that the GP model built from a particular value of $\theta$ is the best performing model. Since GP models are constructed from assumptions about the conditional probability distribution of the data $p(\mathbf{y}|\mathbf{X})$, it is natural to use the negative log-likelihood of the training data as a model selection criterion. 

\begin{align*}
  Q(\theta) & = - log \ p(\mathbf{y}|\mathbf{X}, \theta) \\
            & = \frac{1}{2} \mathbf{y}^\intercal (\mathbf{K}_\theta + \sigma^{2} \mathbf{I})^{-1} \mathbf{y} + \frac{1}{2}|\mathbf{K}_\theta + \sigma^{2} \mathbf{I}| + \frac{N}{2}log(2\pi) \\
  \mathbf{K}_\theta & = [K_{\theta}(\mathbf{x}_i, \mathbf{x}_j)]_{N \times N}
\end{align*}

The model selection problem can now be expressed as the minimization problem shown below.

\begin{align*}
\theta^* = arg\min_{\theta} \ Q(\theta)
\end{align*}

The objective function $Q(\theta)$ can have multiple local minima, and evaluating the value of $Q(.)$ at any given $\theta$ requires inversion of the matrix $\mathbf{K}_\theta + \sigma^{2} \mathbf{I}$ which has a time complexity $O(N^3)$ as noted above. In the interest of saving computational cost, one cannot use exhaustive search through the domain of the hyper-parameters to inform our choice for $\theta$.

The model selection problem has historically received less attention due to a combination of difficulty and numerous design decisions like choice of hyper-parameter optimization techniques, which makes model selection as much of an art as science. 

Choosing a suitable heuristic or algorithm for model selection for GP models is quite often a balancing exercise between the computational limitations of the modeler and the need to explore the hyper-parameter space in a manner that yields a satisfactory predictive model. Some of the techniques used for model selection in the context of GPR include.


\subsubsection*{Grid Search}

This is the simplest procedure that may be applied for selection of GPR models. This routine constructs a grid of values for $\theta$ as the cartesian product of one dimensional grids for each $\theta_i$, evaluate $Q(.)$ at each such grid point and choose the configuration which yields minimum value of $Q(.)$. The advantage of this technique is its simplicity and control over the computational cost of training GP models for practical problems. The computational cost of this procedure scales with the total number of points on the grid, this is controlled by the scales and number of grid points for each hyper-parameter $\theta_i$.

\subsubsection*{Coupled Simulated Annealing}

Introduced in \citet{Xavier-De-Souza2010}, \emph{Coupled Simulated Annealing} (CSA) is a family of optimization techniques which is a generalization of \emph{Simulated Annealing} (SA) \citep{Kirkpatrick671}. The CSA optimization technique can be understood as a set of parallel SA processes which are coupled by their acceptance probabilities.

From an implementation point of view, CSA follows the same procedure as \emph{grid search}, but after evaluation of the objective $Q(.)$ on the grid, each grid point is iteratively mutated in a random walk fashion. This mutation is accepted or rejected according to the new value of $Q(.)$ as well as its value on the other grid points. This procedure is iterated until some stop criterion is reached.

\subsubsection*{Maximum Likelihood} 

This technique, as described in \citet{Rasmussen:2005:GPM:1162254}, is a form of \emph{gradient descent}. It involves starting with an initial guess for $\theta$ and iteratively improving it by calculating the gradient of $Q(.)$ with respect to $\theta$. The gradient of $Q(.)$ with respect to each hyper-parameter $\theta_i$ can be calculated analytically as shown below.

\begin{align*}
\frac{\partial}{\partial \theta_i} log \ p(\mathbf{y}|\mathbf{X}, \theta) & = \frac{1}{2} \mathbf{y}^\intercal K^{-1} \frac{\partial K}{\partial \theta_i} K^{-1} \mathbf{y}  - \frac{1}{2} tr(K^{-1} \frac{\partial K}{\partial \theta_i})\\
 & = \frac{1}{2} tr((\mathbf{\alpha}\mathbf{\alpha}^\intercal - K^{-1}) \frac{\partial K}{\partial \theta_i}) \ \ where \ \mathbf{\alpha} = K^{-1} \mathbf{y}
\end{align*}


Although gradient based \emph{Maximum Likelihood} looks very similar to the \emph{Gradient Descent} used to learn parametric regression models like \emph{Feed forward Neural Networks}, in the GPR model selection context it introduces an extra computational cost of calculating the gradient of $Q(\theta)$ with respect to each $\theta_i$ in every iteration. Applying this method can sometimes lead to overfitting of the GPR model to the training data \citep{Rasmussen:2005:GPM:1162254}.




\bibliography{references}

\end{document}