\documentclass{article}
\begin{document}

\title{Response Reviewer 1}
\maketitle

The author would like to thank the reviewer for the various comments
and suggestions which greatly helped in improving the quality of the manuscript. 

In the proceeding section they will find the revisions/responses for
each concern outlined. The tracked changes are highlighted in the file \emph{differences.pdf}.

\section*{Draft Comments \& Revisions}

\begin{enumerate}

\item{

\fbox{%
    \parbox{\textwidth}{%
\textbf{Comment}: The abstract assumes that the readers are already introduced to the GP methodology from previous chapters.}
}

\textbf{Response}: I concur with the assertion, given the scope and
organisation of the text it is not an unreasonable assumption. The
chapter nevertheless strives to give a bare minimum theoretical
treatment of Gaussian Process modelelling.

}


\item{
  \fbox{%
    \parbox{\textwidth}{%
\textbf{Comment}: Replace $K_p$ with Kp}
}

\textbf{Response}: Appropriate changes made in the manuscript.

}

\item{
  
\fbox{%
\parbox{\textwidth}{%
\textbf{Comment}: Section 1, last paragraph. The SPDF is a repository
for a wide variety of data sources from spacecrafts and ground-based
instruments. It is not intended to forecast geomagnetic indices. It is
intended for research and validation purposes only.
}
} 


\textbf{Response}: Appropriate changes made in section 1, page 2.

}

\item{

\fbox{%
\parbox{\textwidth}{%

\textbf{Comment}: Section 3. Can the author point to exactly why the
Dst is chosen for this work as opposed to some other ground-based
index? Perhaps also identify other tools and techniques for the
prediction of other indices similar to the Dst. Also “state of the
art” could be better said “other forecast alogrithms/models”. Some of
the references are obsolete while many of them are unverified and have
not transitioned to forecast models. The notable ones are the Wing et
al. 2005 and Bala et al. 2009 which are operational in near-real
time. Wing et al. 2005 was picked by NOAA SWPC for operational purpose.

}
} 


\textbf{Response}: Since scope of the chapter is to illustrate how to
apply GP methodology for space weather forecasting, the particular
choice of geomagnetic index and its justification is tangential to the
primary aim. That being said, it is evident from the introduction that
the Dst index being a measure of geomagnetic disturbance strength is
an important quantity and thus an easy choice for a case study.

The changes with respect to forecast algorithms has been made on page
3, section 2.1.
}

\item{

\fbox{%
\parbox{\textwidth}{%
\textbf{Comment}: Section 3.2: more examples of probabilistic
forecasting models in the space weather context would be relevant here.

}
} 


\textbf{Response}: Changes made on page 3, section 2.2.


}


\item{

\fbox{%
\parbox{\textwidth}{%
\textbf{Comment}: Section 4... Are there references in the space
weather literature with applications involving this technique?
}
} 


\textbf{Response}: A reference using the current technique has been
added on page 4, section 4.


}


\item{

\fbox{%
\parbox{\textwidth}{

\textbf{Comment}: Section 4.1: Given this will cover a broader
audience, a sentence or two introducing the GP to the ML context would
be relevant. Perhaps saying it as “this is a kernel-based, supervised
learning approach that is good for prediction tasks” or such could
make a nice segue into the section.

}
} 


\textbf{Response}: Gaussian Processes have a rich theory intersecting
areas of probability theory, functional analysis, linear
operators and positive type functions. Because the multiple theoretical
perspectives that can be used to introduce them, there is a chance of
creating confusion for new practitioners especially those with less 
mathematical experience. Thus it was deemed non-essential to mention
terms such as "kernel-based, supervised learning approaches". The
chapter nevertheless gives a basic introduction to kernel functions
and their characteristics.


}

\item{
    \fbox{
      \parbox{\textwidth}{
        \textbf{Comment}: Section 5: define OSA.
$X_{t-1}$ having composed of the predicted index in the formulation can be
precarious and can have some unintended consequences as a result. Dst
is a slow response index to the changing IMF conditions. Applying time
histories of the Dst index along with the prevailing IMF parameters
will be an interesting prediction model at the outset- two causally
coupled quantities with a large time lag (>1 hr) as inputs. It would
be interesting to see a mere persistence model like the GPR comparing
with this. It will also be nice to see a model that only takes the SW
as inputs to predict the Dst index.

        
      }
    }
    \textbf{Response}: The persistence model has been compared with
    the GP based models in table 5 on page 23. Due to strong
    persistence behavior, the author believes that a model based
    solely on solar wind inputs would not give attrative
    performance. For that reason and for sake of brevity it has been
    omitted from the chapter.

}


\item{
    \fbox{
      \parbox{\textwidth}{
        \textbf{Comment}: Section 5.2: The author has chosen Vsw and
        Bz to augment their original AR model. In predicting the Dst
        index it is useful to bring in the dynamic pressure term or at
        a minimum the density term from the SW parameters. Did the
        author try to model that? It will be good to see the density
        term added in their or a mention of why it was omitted. A
        clarification is needed here.

        
      }
    }
    \textbf{Response}: Modifications made on page 6, section 4.4.

}

\item{
    \fbox{
      \parbox{\textwidth}{
        \textbf{Comment}: Section 7.4, It is not clear how the GPARX
        model number is chosen from the conditions shown here: 3 <= p
        + pv + pb <= 8. However the validation shows that this is not
        preserved. Can the author explain this?


      }
    }
    \textbf{Response}: Modifications made on page 12, section 6.

}

\item{
    \fbox{
      \parbox{\textwidth}{
        \textbf{Comment}: Figures 5 and 6 tend to indicate the problem
        of persistence alluded to in my comment 8 above. It looks
        (hard to tell from a b/w figure though) as if the prediction
        is following the actual Dst. If this is indeed the case (again
        please clarify this with a better plot or with color), the
        model clearly misses the onset which is critical to a good
        forecast model.

      }
    }
    \textbf{Response}: Figures 5, 6 and 7 have been modified, also
    effort was made to include a strong geomagnetic event in them.

}

\item{
    \fbox{
      \parbox{\textwidth}{
        \textbf{Comment}: How does the results here compare to other
        models of similar technique? Can the author also talk about
        noise free observations if there is relevance here?


      }
    }
    \textbf{Response}: Since the noise component in the model has
    small variance, it is very close to the noise free paradigm. Since
    is difficult to obtain codes for various space weather models and
    since it is not the aim of the chapter to compare forecast models,
    comparisons with other methods have been avoided. It is also worth
    noting that a very large number of Dst forecast models are
    deterministic forecasts and not probabilistic.

}

\item{
    \fbox{
      \parbox{\textwidth}{
        \textbf{Comment}: Even though it may not be relevant here it
        might be good to talk about the operational scalability of
        this model to a real-time framework and any other key
        considerations from a forecaster perspective.


      }
    }
    \textbf{Response}: Although operational scalability of the
    technique is not of relevance and not explicitly discussed,
    on page 8 the reader is informed on the time complexity of GP
    based inference.

}

\item{
    \fbox{
      \parbox{\textwidth}{
        \textbf{Comment}: The title “many hours ahead prediction..”,
        at least in my view, is slightly misleading. Generally this
        refers to long-term predictions. But I find this as an
        artifact of the GP process which only comes with familiarity
        of the process itself.

      }
    }
    \textbf{Response}: The title "many hours ahead prediction .." was
    a working title and has been modified when submitting the first
    draft of the chapter. Due to the online editorial system not being
    able to reflect that modification there was a
    miscommunication. The revised draft has an accurate chapter title.

}

\end{enumerate}


\end{document}